%%%%%%%%%%%%%%%%%%%%%%%%%%%%%%%%%%%%%%%%%%%%%%%%%%%%%%%%%%%%%%%%%%%%%%%%%%%%%%%
%% Name:        fdrepdlg.tex
%% Purpose:     wxFindReplaceDialog documentation
%% Author:      Vadim Zeitlin
%% Modified by:
%% Created:     01.08.01
%% RCS-ID:      $Id: fdrepdlg.tex,v 1.1.1.1 2009/10/09 02:55:29 jack Exp $
%% Copyright:   (c) 2001 Vadim Zeitlin
%% License:     wxWindows license
%%%%%%%%%%%%%%%%%%%%%%%%%%%%%%%%%%%%%%%%%%%%%%%%%%%%%%%%%%%%%%%%%%%%%%%%%%%%%%%

\section{\class{wxFindDialogEvent}}\label{wxfinddialogevent}

wxFindReplaceDialog events

\wxheading{Derived from}

\helpref{wxCommandEvent}{wxcommandevent}

\wxheading{Include files}

<wx/fdrepdlg.h>

\wxheading{Event table macros}

To process a command event from 
\helpref{wxFindReplaceDialog}{wxfindreplacedialog}, use these event handler
macros to direct input to member functions that take a wxFindDialogEvent
argument. The {\it id} parameter is the identifier of the find dialog and you
may usually specify $-1$ for it unless you plan to have several find dialogs
sending events to the same owner window simultaneously.

\twocolwidtha{7cm}
\begin{twocollist}\itemsep=0pt
\twocolitem{{\bf EVT\_FIND(id, func)}}{Find button was pressed in the dialog.}
\twocolitem{{\bf EVT\_FIND\_NEXT(id, func)}}{Find next button was pressed in the dialog.}
\twocolitem{{\bf EVT\_FIND\_REPLACE(id, func)}}{Replace button was pressed in the dialog.}
\twocolitem{{\bf EVT\_FIND\_REPLACE\_ALL(id, func)}}{Replace all button was pressed in the dialog.}
\twocolitem{{\bf EVT\_FIND\_CLOSE(id, func)}}{The dialog is being destroyed,
any pointers to it cannot be used any longer.}
\end{twocollist}%

\latexignore{\rtfignore{\wxheading{Members}}}

\membersection{wxFindDialogEvent::wxFindDialogEvent}\label{wxfinddialogeventwxfinddialogevent}

\func{}{wxFindDialogEvent}{\param{wxEventType }{commandType = wxEVT\_NULL}, \param{int }{id = 0}}

Constuctor used by wxWidgets only.

\membersection{wxFindDialogEvent::GetFlags}\label{wxfinddialogeventgetflags}

\constfunc{int}{GetFlags}{\void}

Get the currently selected flags: this is the combination of {\tt wxFR\_DOWN},
{\tt wxFR\_WHOLEWORD} and {\tt wxFR\_MATCHCASE} flags.

\membersection{wxFindDialogEvent::GetFindString}\label{wxfinddialogeventgetfindstring}

\constfunc{wxString}{GetFindString}{\void}

Return the string to find (never empty).

\membersection{wxFindDialogEvent::GetReplaceString}\label{wxfinddialogeventgetreplacestring}

\constfunc{const wxString\&}{GetReplaceString}{\void}

Return the string to replace the search string with (only for replace and
replace all events).

\membersection{wxFindDialogEvent::GetDialog}\label{wxfinddialogeventgetdialog}

\constfunc{wxFindReplaceDialog*}{GetDialog}{\void}

Return the pointer to the dialog which generated this event.

\section{\class{wxFindReplaceData}}\label{wxfindreplacedata}

wxFindReplaceData holds the data for 
\helpref{wxFindReplaceDialog}{wxfindreplacedialog}. It is used to initialize
the dialog with the default values and will keep the last values from the
dialog when it is closed. It is also updated each time a 
\helpref{wxFindDialogEvent}{wxfinddialogevent} is generated so instead of
using the wxFindDialogEvent methods you can also directly query this object.

Note that all {\tt SetXXX()} methods may only be called before showing the
dialog and calling them has no effect later.

\wxheading{Include files}

\begin{verbatim}
#include <wx/fdrepdlg.h>
\end{verbatim}

\wxheading{Derived from}

\helpref{wxObject}{wxobject}

\wxheading{Data structures}

Flags used by 
\helpref{wxFindReplaceData::GetFlags()}{wxfindreplacedatagetflags} and
\helpref{wxFindDialogEvent::GetFlags()}{wxfinddialogeventgetflags}:

\begin{verbatim}
enum wxFindReplaceFlags
{
    // downward search/replace selected (otherwise - upwards)
    wxFR_DOWN       = 1,

    // whole word search/replace selected
    wxFR_WHOLEWORD  = 2,

    // case sensitive search/replace selected (otherwise - case insensitive)
    wxFR_MATCHCASE  = 4
}
\end{verbatim}

These flags can be specified in 
\helpref{wxFindReplaceDialog ctor}{wxfindreplacedialogctor} or 
\helpref{Create()}{wxfindreplacedialogcreate}:

\begin{verbatim}
enum wxFindReplaceDialogStyles
{
    // replace dialog (otherwise find dialog)
    wxFR_REPLACEDIALOG = 1,

    // don't allow changing the search direction
    wxFR_NOUPDOWN      = 2,

    // don't allow case sensitive searching
    wxFR_NOMATCHCASE   = 4,

    // don't allow whole word searching
    wxFR_NOWHOLEWORD   = 8
}
\end{verbatim}

\latexignore{\rtfignore{\wxheading{Members}}}

\membersection{wxFindReplaceData::wxFindReplaceData}\label{wxfindreplacedatactor}

\func{}{wxFindReplaceData}{\param{wxUint32 }{flags = 0}}

Constuctor initializes the flags to default value ($0$).

\membersection{wxFindReplaceData::GetFindString}\label{wxfindreplacedatagetfindstring}

\func{const wxString\&}{GetFindString}{\void}

Get the string to find.

\membersection{wxFindReplaceData::GetReplaceString}\label{wxfindreplacedatagetreplacestring}

\func{const wxString\&}{GetReplaceString}{\void}

Get the replacement string.

\membersection{wxFindReplaceData::GetFlags}\label{wxfindreplacedatagetflags}

\constfunc{int}{GetFlags}{\void}

Get the combination of {\tt wxFindReplaceFlags} values.

\membersection{wxFindReplaceData::SetFlags}\label{wxfindreplacedatasetflags}

\func{void}{SetFlags}{\param{wxUint32 }{flags}}

Set the flags to use to initialize the controls of the dialog.

\membersection{wxFindReplaceData::SetFindString}\label{wxfindreplacedatasetfindstring}

\func{void}{SetFindString}{\param{const wxString\& }{str}}

Set the string to find (used as initial value by the dialog).

\membersection{wxFindReplaceData::SetReplaceString}\label{wxfindreplacedatasetreplacestring}

\func{void}{SetReplaceString}{\param{const wxString\& }{str}}

Set the replacement string (used as initial value by the dialog).

\section{\class{wxFindReplaceDialog}}\label{wxfindreplacedialog}

wxFindReplaceDialog is a standard modeless dialog which is used to allow the
user to search for some text (and possibly replace it with something else).
The actual searching is supposed to be done in the owner window which is the
parent of this dialog. Note that it means that unlike for the other standard
dialogs this one {\bf must} have a parent window. Also note that there is no
way to use this dialog in a modal way; it is always, by design and
implementation, modeless.

Please see the dialogs sample for an example of using it.

\wxheading{Include files}

\begin{verbatim}
#include <wx/fdrepdlg.h>
\end{verbatim}

\wxheading{Derived from}

\helpref{wxDialog}{wxdialog}

\latexignore{\rtfignore{\wxheading{Members}}}

\membersection{wxFindReplaceDialog::wxFindReplaceDialog}\label{wxfindreplacedialogctor}

\func{}{wxFindReplaceDialog}{\void}

\func{}{wxFindReplaceDialog}{\param{wxWindow * }{parent}, \param{wxFindReplaceData* }{data}, \param{const wxString\& }{title}, \param{int }{style = 0}}

After using default constructor \helpref{Create()}{wxfindreplacedialogcreate} 
must be called.

The {\it parent} and {\it data} parameters must be non-{\tt NULL}.

\membersection{wxFindReplaceDialog::\destruct{wxFindReplaceDialog}}\label{wxfindreplacedialogdtor}

\func{}{\destruct{wxFindReplaceDialog}}{\void}

Destructor.

\membersection{wxFindReplaceDialog::Create}\label{wxfindreplacedialogcreate}

\func{bool}{Create}{\param{wxWindow * }{parent}, \param{wxFindReplaceData* }{data}, \param{const wxString\& }{title}, \param{int }{style = 0}}

Creates the dialog; use \helpref{Show}{wxwindowshow} to show it on screen.

The {\it parent} and {\it data} parameters must be non-{\tt NULL}.
\membersection{wxFindReplaceDialog::GetData}\label{wxfindreplacedialoggetdata}

\constfunc{const wxFindReplaceData*}{GetData}{\void}

Get the \helpref{wxFindReplaceData}{wxfindreplacedata} object used by this
dialog.

