%%%%%%%%%%%%%%%%%%%%%%%%%%%%%%%%%%%%%%%%%%%%%%%%%%%%%%%%%%%%%%%%%%%%%%%%%%%%%%%
%% Name:        longlong.tex
%% Purpose:     wxLongLong documentation
%% Author:      Vadim Zeitlin
%% Modified by:
%% Created:     07.03.00
%% RCS-ID:      $Id: longlong.tex,v 1.1.1.1 2009/10/09 02:55:36 jack Exp $
%% Copyright:   (c) Vadim Zeitlin
%% License:     wxWindows license
%%%%%%%%%%%%%%%%%%%%%%%%%%%%%%%%%%%%%%%%%%%%%%%%%%%%%%%%%%%%%%%%%%%%%%%%%%%%%%%

\section{\class{wxLongLong}}\label{wxlonglong}

This class represents a signed 64 bit long number. It is implemented using the
native 64 bit type where available (machines with 64 bit longs or compilers
which have (an analog of) {\it long long} type) and uses the emulation code in
the other cases which ensures that it is the most efficient solution for
working with 64 bit integers independently of the architecture.

wxLongLong defines all usual arithmetic operations such as addition,
subtraction, bitwise shifts and logical operations as well as multiplication
and division (not yet for the machines without native {\it long long}). It
also has operators for implicit construction from and conversion to the native 
{\it long long} type if it exists and {\it long}.

You would usually use this type in exactly the same manner as any other
(built-in) arithmetic type. Note that wxLongLong is a signed type, if you
want unsigned values use wxULongLong which has exactly the same API as
wxLongLong except when explicitly mentioned otherwise.

If a native (i.e. supported directly by the compiler) 64 bit integer type was
found to exist, {\it wxLongLong\_t} macro will be defined to correspond to it.
Also, in this case only, two additional macros will be defined: 
\helpref{wxLongLongFmtSpec}{wxlonglongfmtspec} for printing 64 bit integers
using the standard {\tt printf()} function (but see also 
\helpref{ToString()}{wxlonglongtostring} for a more portable solution) and
\helpref{wxLL}{wxll} for defining 64 bit integer compile-time constants.

\wxheading{Derived from}

No base class

\wxheading{Include files}

<wx/longlong.h>

\latexignore{\rtfignore{\wxheading{Members}}}


\membersection{wxLongLong::wxLongLong}\label{wxlonglongwxlonglongdef}

\func{}{wxLongLong}{\void}

Default constructor initializes the object to 0.


\membersection{wxLongLong::wxLongLong}\label{wxlonglongwxlonglongll}

\func{}{wxLongLong}{\param{wxLongLong\_t }{ll}}

Constructor from native long long (only for compilers supporting it).


\membersection{wxLongLong::wxLongLong}\label{wxlonglongwxlonglong}

\func{}{wxLongLong}{\param{long }{hi}, \param{unsigned long }{lo}}

Constructor from 2 longs: the high and low part are combined into one
wxLongLong.


\membersection{wxLongLong::operator=}\label{wxlonglongoperatorassign}

\func{wxLongLong\& operator}{operator=}{\param{wxLongLong\_t }{ll}}

Assignment operator from native long long (only for compilers supporting it).


\membersection{wxLongLong::operator=}\label{wxlonglongoperatorassignull}

\func{wxLongLong\& operator}{operator=}{\param{wxULongLong\_t }{ll}}

Assignment operator from native unsigned long long (only for compilers 
supporting it).

\newsince{2.7.0}

\membersection{wxLongLong::operator=}\label{wxlonglongoperatorassignlong}

\func{wxLongLong\& operator}{operator=}{\param{long }{l}}

Assignment operator from long.

\newsince{2.7.0}

\membersection{wxLongLong::operator=}\label{wxlonglongoperatorassignulong}

\func{wxLongLong\& operator}{operator=}{\param{unsigned long }{l}}

Assignment operator from unsigned long.

\newsince{2.7.0}

\membersection{wxLongLong::operator=}\label{wxlonglongoperatorassignulonglong}

\func{wxLongLong\& operator}{operator=}{\param{const wxULongLong \& }{ll}}

Assignment operator from unsigned long long. The sign bit will be copied too.

\newsince{2.7.0}

\membersection{wxLongLong::Abs}\label{wxlonglongabs}

\constfunc{wxLongLong}{Abs}{\void}

\func{wxLongLong\&}{Abs}{\void}

Returns an absolute value of wxLongLong - either making a copy (const version)
or modifying it in place (the second one).  Not in wxULongLong.


\membersection{wxLongLong::Assign}\label{wxlonglongassign}

\func{wxLongLong\&}{Assign}{\param{double }{d}}

This allows to convert a double value to wxLongLong type. Such conversion is
not always possible in which case the result will be silently truncated in a
platform-dependent way.  Not in wxULongLong.


\membersection{wxLongLong::GetHi}\label{wxlonglonggethi}

\constfunc{long}{GetHi}{\void}

Returns the high 32 bits of 64 bit integer.


\membersection{wxLongLong::GetLo}\label{wxlonglonggetlo}

\constfunc{unsigned long}{GetLo}{\void}

Returns the low 32 bits of 64 bit integer.


\membersection{wxLongLong::GetValue}\label{wxlonglonggetvalue}

\constfunc{wxLongLong\_t}{GetValue}{\void}

Convert to native long long (only for compilers supporting it)


\membersection{wxLongLong::ToDouble}\label{wxlonglonggetdouble}

\constfunc{double}{ToDouble}{\void}

Returns the value as \texttt{double}.


\membersection{wxLongLong::ToLong}\label{wxlonglongtolong}

\constfunc{long}{ToLong}{\void}

Truncate wxLongLong to long. If the conversion loses data (i.e. the wxLongLong
value is outside the range of built-in long type), an assert will be triggered
in debug mode.


\membersection{wxLongLong::ToString}\label{wxlonglongtostring}

\constfunc{wxString}{ToString}{\void}

Returns the string representation of a wxLongLong.


\membersection{wxLongLong::operator$+$}\label{wxlonglongoperatorplus}

\constfunc{wxLongLong}{operator$+$}{\param{const wxLongLong\& }{ll}}

Adds 2 wxLongLongs together and returns the result.


\membersection{wxLongLong::operator$+=$}\label{wxlonglongoperatorplusassign}

\func{wxLongLong\&}{operator+}{\param{const wxLongLong\& }{ll}}

Add another wxLongLong to this one.


\membersection{wxLongLong::operator$++$}\label{wxlonglongoperatorinc}

\func{wxLongLong\&}{operator$++$}{\void}

\func{wxLongLong\&}{operator$++$}{\param{int}{}}

Pre/post increment operator.


\membersection{wxLongLong::operator$-$}\label{wxlonglongoperatorunaryminus}

\constfunc{wxLongLong}{operator$-$}{\void}

Returns the value of this wxLongLong with opposite sign.  Not in wxULongLong.


\membersection{wxLongLong::operator$-$}\label{wxlonglongoperatorminus}

\constfunc{wxLongLong}{operator$-$}{\param{const wxLongLong\& }{ll}}

Subtracts 2 wxLongLongs and returns the result.


\membersection{wxLongLong::operator$-=$}\label{wxlonglongoperatorminusassign}

\func{wxLongLong\&}{operator-}{\param{const wxLongLong\& }{ll}}

Subtracts another wxLongLong from this one.


\membersection{wxLongLong::operator$--$}\label{wxlonglongoperatordec}

\func{wxLongLong\&}{operator$--$}{\void}

\func{wxLongLong\&}{operator$--$}{\param{int}{}}

Pre/post decrement operator.

% TODO document all other arithmetics operations: shifts, multiplication,
% division, bitwise, comparison

