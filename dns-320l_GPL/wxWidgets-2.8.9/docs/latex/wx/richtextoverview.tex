\section{wxRichTextCtrl overview}\label{wxrichtextctrloverview}

{\bf Major classes:} \helpref{wxRichTextCtrl}{wxrichtextctrl}, \helpref{wxRichTextBuffer}{wxrichtextbuffer}, \helpref{wxRichTextEvent}{wxrichtextevent}

{\bf Helper classes:} \helpref{wxRichTextAttr}{wxrichtextattr}, \helpref{wxTextAttrEx}{wxtextattrex}, 
\helpref{wxRichTextRange}{wxrichtextrange}

{\bf File handler classes:} \helpref{wxRichTextFileHandler}{wxrichtextfilehandler}, \helpref{wxRichTextHTMLHandler}{wxrichtexthtmlhandler}, 
\helpref{wxRichTextXMLHandler}{wxrichtextxmlhandler}

{\bf Style classes:} \helpref{wxRichTextCharacterStyleDefinition}{wxrichtextcharacterstyledefinition}, 
\helpref{wxRichTextParagraphStyleDefinition}{wxrichtextparagraphstyledefinition}, 
\helpref{wxRichTextListStyleDefinition}{wxrichtextliststyledefinition}, 
\helpref{wxRichTextStyleSheet}{wxrichtextstylesheet}

{\bf Additional controls:} \helpref{wxRichTextStyleComboCtrl}{wxrichtextstylecomboctrl}, 
\helpref{wxRichTextStyleListBox}{wxrichtextstylelistbox}, 
\helpref{wxRichTextStyleListCtrl}{wxrichtextstylelistctrl}

{\bf Printing classes:} \helpref{wxRichTextPrinting}{wxrichtextprinting}, 
\helpref{wxRichTextPrintout}{wxrichtextprintout}, 
\helpref{wxRichTextHeaderFooterData}{wxrichtextheaderfooterdata}

{\bf Dialog classes:} \helpref{wxRichTextStyleOrganiserDialog}{wxrichtextstyleorganiserdialog}, 
\helpref{wxRichTextFormattingDialog}{wxrichtextformattingdialog}, 
\helpref{wxSymbolPickerDialog}{wxsymbolpickerdialog}

wxRichTextCtrl provides a generic implementation of a rich text editor that can handle different character
styles, paragraph formatting, and images. It's aimed at editing 'natural' language text - if you need an editor
that supports code editing, wxStyledTextCtrl is a better choice.

Despite its name, it cannot currently read or write RTF (rich text format) files. Instead, it
uses its own XML format, and can also read and write plain text. In future we expect to provide
RTF file capabilities. Custom file formats can be supported by creating additional
file handlers and registering them with the control.

wxRichTextCtrl is largely compatible with the wxTextCtrl API, but extends it where necessary.
The control can be used where the native rich text capabilities of wxTextCtrl are not
adequate (this is particularly true on Windows) and where more direct access to
the content representation is required. It is difficult and inefficient to read
the style information in a wxTextCtrl, whereas this information is readily
available in wxRichTextCtrl. Since it's written in pure wxWidgets, any customizations
you make to wxRichTextCtrl will be reflected on all platforms.

wxRichTextCtrl supports basic printing via the easy-to-use \helpref{wxRichTextPrinting}{wxrichtextprinting} class.
Creating applications with simple word processing features is simplified with the inclusion of\rtfsp
\helpref{wxRichTextFormattingDialog}{wxrichtextformattingdialog}, a tabbed dialog allowing
interactive tailoring of paragraph and character styling. Also provided is the multi-purpose dialog\rtfsp
\helpref{wxRichTextStyleOrganiserDialog}{wxrichtextstyleorganiserdialog} that can be used for
managing style definitions, browsing styles and applying them, or selecting list styles with
a renumber option.

There are a few disadvantages to using wxRichTextCtrl. It is not native,
so does not behave exactly as a native wxTextCtrl, although common editing conventions
are followed. Users may miss the built-in spelling correction on Mac OS X, or any
special character input that may be provided by the native control. It would also
be a poor choice if intended users rely on screen readers that would be not work well
with non-native text input implementation. You might mitigate this by providing
the choice between wxTextCtrl and wxRichTextCtrl, with fewer features in the
former case.

A good way to understand wxRichTextCtrl's capabilities is to compile and run the
sample, {\tt samples/richtext}, and browse the code. The following screenshot shows the sample in action:

$$\image{8cm;0cm}{richtextctrl.gif}$$

\wxheading{Example}\label{wxrichtextctrlexample}

The following code is taken from the sample, and adds text and styles to a rich text control programmatically.

{\small
\begin{verbatim}
    wxRichTextCtrl* richTextCtrl = new wxRichTextCtrl(splitter, wxID_ANY, wxEmptyString, wxDefaultPosition, wxSize(200, 200), wxVSCROLL|wxHSCROLL|wxNO_BORDER|wxWANTS_CHARS);

    wxFont textFont = wxFont(12, wxROMAN, wxNORMAL, wxNORMAL);
    wxFont boldFont = wxFont(12, wxROMAN, wxNORMAL, wxBOLD);
    wxFont italicFont = wxFont(12, wxROMAN, wxITALIC, wxNORMAL);

    wxFont font(12, wxROMAN, wxNORMAL, wxNORMAL);

    m_richTextCtrl->SetFont(font);

    wxRichTextCtrl& r = richTextCtrl;

    r.BeginSuppressUndo();

    r.BeginParagraphSpacing(0, 20);

    r.BeginAlignment(wxTEXT_ALIGNMENT_CENTRE);
    r.BeginBold();

    r.BeginFontSize(14);
    r.WriteText(wxT("Welcome to wxRichTextCtrl, a wxWidgets control for editing and presenting styled text and images"));
    r.EndFontSize();
    r.Newline();

    r.BeginItalic();
    r.WriteText(wxT("by Julian Smart"));
    r.EndItalic();

    r.EndBold();

    r.Newline();
    r.WriteImage(wxBitmap(zebra_xpm));

    r.EndAlignment();

    r.Newline();
    r.Newline();

    r.WriteText(wxT("What can you do with this thing? "));
    r.WriteImage(wxBitmap(smiley_xpm));
    r.WriteText(wxT(" Well, you can change text "));

    r.BeginTextColour(wxColour(255, 0, 0));
    r.WriteText(wxT("colour, like this red bit."));
    r.EndTextColour();

    r.BeginTextColour(wxColour(0, 0, 255));
    r.WriteText(wxT(" And this blue bit."));
    r.EndTextColour();

    r.WriteText(wxT(" Naturally you can make things "));
    r.BeginBold();
    r.WriteText(wxT("bold "));
    r.EndBold();
    r.BeginItalic();
    r.WriteText(wxT("or italic "));
    r.EndItalic();
    r.BeginUnderline();
    r.WriteText(wxT("or underlined."));
    r.EndUnderline();

    r.BeginFontSize(14);
    r.WriteText(wxT(" Different font sizes on the same line is allowed, too."));
    r.EndFontSize();

    r.WriteText(wxT(" Next we'll show an indented paragraph."));

    r.BeginLeftIndent(60);
    r.Newline();

    r.WriteText(wxT("Indented paragraph."));
    r.EndLeftIndent();

    r.Newline();

    r.WriteText(wxT("Next, we'll show a first-line indent, achieved using BeginLeftIndent(100, -40)."));

    r.BeginLeftIndent(100, -40);
    r.Newline();

    r.WriteText(wxT("It was in January, the most down-trodden month of an Edinburgh winter."));
    r.EndLeftIndent();

    r.Newline();

    r.WriteText(wxT("Numbered bullets are possible, again using subindents:"));

    r.BeginNumberedBullet(1, 100, 60);
    r.Newline();

    r.WriteText(wxT("This is my first item. Note that wxRichTextCtrl doesn't automatically do numbering, but this will be added later."));
    r.EndNumberedBullet();

    r.BeginNumberedBullet(2, 100, 60);
    r.Newline();

    r.WriteText(wxT("This is my second item."));
    r.EndNumberedBullet();

    r.Newline();

    r.WriteText(wxT("The following paragraph is right-indented:"));

    r.BeginRightIndent(200);
    r.Newline();

    r.WriteText(wxT("It was in January, the most down-trodden month of an Edinburgh winter. An attractive woman came into the cafe, which is nothing remarkable."));
    r.EndRightIndent();

    r.Newline();

    wxArrayInt tabs;
    tabs.Add(400);
    tabs.Add(600);
    tabs.Add(800);
    tabs.Add(1000);
    wxTextAttrEx attr;
    attr.SetFlags(wxTEXT_ATTR_TABS);
    attr.SetTabs(tabs);
    r.SetDefaultStyle(attr);
    
    r.WriteText(wxT("This line contains tabs:\tFirst tab\tSecond tab\tThird tab"));

    r.Newline();
    r.WriteText(wxT("Other notable features of wxRichTextCtrl include:"));

    r.BeginSymbolBullet(wxT('*'), 100, 60);
    r.Newline();
    r.WriteText(wxT("Compatibility with wxTextCtrl API"));
    r.EndSymbolBullet();

    r.WriteText(wxT("Note: this sample content was generated programmatically from within the MyFrame constructor in the demo. The images were loaded from inline XPMs. Enjoy wxRichTextCtrl!"));

    r.EndSuppressUndo();
\end{verbatim}
}

\subsection{Programming with wxRichTextCtrl}

\subsubsection{Starting to use wxRichTextCtrl}

You need to include {\tt <wx/richtext/richtextctrl.h>} in your source, and link
with the appropriate wxWidgets library with {\tt richtext} suffix. Put the rich text
library first in your link line to avoid unresolved symbols.

Then you can create a wxRichTextCtrl, with the wxWANT\_CHARS style if you want tabs to
be processed by the control rather than being used for navigation between controls.

\subsubsection{wxRichTextCtrl and styles}

Styling attributes are represented by three classes: \helpref{wxTextAttr}{wxtextattr}, \helpref{wxTextAttrEx}{wxtextattrex} and \helpref{wxRichTextAttr}{wxrichtextattr}.
wxTextAttr is shared across all controls that are derived from wxTextCtrlBase and
can store basic character and paragraph attributes. wxTextAttrEx derives
from wxTextAttr and adds some further attributes that are only supported
by wxRichTextCtrl. Finally, wxRichTextAttr is a more efficient version
of wxTextAttrEx that doesn't use a wxFont object and can be used to
query styles more quickly. wxTextAttrEx and wxRichTextAttr are largely
interchangeable and have suitable conversion operators between them.

When setting a style, the flags of the attribute object determine which
attributes are applied. When querying a style, the passed flags are ignored
except (optionally) to determine whether attributes should be retrieved from
character content or from the paragraph object.

wxRichTextCtrl takes a layered approach to styles, so that different parts of
the content may be responsible for contributing different attributes to the final
style you see on the screen.

There are four main notions of style within a control:

\begin{enumerate}\itemsep=0pt
\item {\bf Basic style:} the fundamental style of a control, onto which any other
styles are layered. It provides default attributes, and changing the basic style
may immediately change the look of the content depending on what other styles
the content uses. Calling wxRichTextCtrl::SetFont changes the font for the basic style.
The basic style is set with \helpref{wxRichTextCtrl::SetBasicStyle}{wxrichtextctrlsetbasicstyle}.
\item {\bf Paragraph style:} each paragraph has attributes that are set independently
from other paragraphs and independently from the content within the paragraph.
Normally, these attributes are paragraph-related, such as alignment and indentation,
but it is possible to set character attributes too.
The paragraph style can be set independently of its content by passing wxRICHTEXT\_SETSTYLE\_PARAGRAPHS\_ONLY
to \helpref{wxRichTextCtrl::SetStyleEx}{wxrichtextctrlsetstyleex}.
\item {\bf Character style:} characters within each paragraph can have attributes.
A single character, or a run of characters, can have a particular set of attributes.
The character style can be with \helpref{wxRichTextCtrl::SetStyle}{wxrichtextctrlsetstyle} or 
\helpref{wxRichTextCtrl::SetStyleEx}{wxrichtextctrlsetstyleex}.
\item {\bf Default style:} this is the `current' style that determines the
style of content that is subsequently typed, pasted or programmatically inserted.
The default style is set with \helpref{wxRichTextCtrl::SetDefaultStyle}{wxrichtextctrlsetdefaultstyle}.
\end{enumerate}

What you see on the screen is the dynamically {\it combined} style, found by merging
the first three of the above style types (the fourth is only a guide for future content
insertion and therefore does not affect the currently displayed content).

To make all this more concrete, here are examples of where you might set these different
styles:

\begin{enumerate}\itemsep=0pt
\item You might set the {\bf basic style} to have a Times Roman font in 12 point,
left-aligned, with two millimetres of spacing after each paragraph.
\item You might set the {\bf paragraph style} (for one particular paragraph) to
be centred.
\item You might set the {\bf character style} of one particular word to bold.
\item You might set the {\bf default style} to be underlined, for subsequent
inserted text.
\end{enumerate}

Naturally you can do any of these things either using your own UI, or programmatically.

The basic wxTextCtrl doesn't make the same distinctions as wxRichTextCtrl regarding
attribute storage. So we need finer control when setting and retrieving
attributes. \helpref{wxRichTextCtrl::SetStyleEx}{wxrichtextctrlsetstyleex} takes a {\it flags} parameter:

\begin{itemize}\itemsep=0pt
\item wxRICHTEXT\_SETSTYLE\_OPTIMIZE specifies that the style should be changed only if
the combined attributes are different from the attributes for the current object. This is important when
applying styling that has been edited by the user, because he has just edited the {\it combined} (visible)
style, and wxRichTextCtrl wants to leave unchanged attributes associated with their original objects
instead of applying them to both paragraph and content objects.
\item wxRICHTEXT\_SETSTYLE\_PARAGRAPHS\_ONLY specifies that only paragraph objects within the given range
should take on the attributes.
\item wxRICHTEXT\_SETSTYLE\_CHARACTERS\_ONLY specifies that only content objects (text or images) within the given range
should take on the attributes.
\item wxRICHTEXT\_SETSTYLE\_WITH\_UNDO specifies that the operation should be undoable.
\end{itemize}

It's great to be able to change arbitrary attributes in a wxRichTextCtrl, but
it can be unwieldy for the user or programmer to set attributes separately. Word processors have collections
of styles that you can tailor or use as-is, and this means that you can set a heading with one click
instead of marking text in bold, specifying a large font size, and applying a certain
paragraph spacing and alignment for every such heading. Similarly,
wxWidgets provides a class called \helpref{wxRichTextStyleSheet}{wxrichtextstylesheet} which manages style definitions
(\helpref{wxRichTextParagraphStyleDefinition}{wxrichtextparagraphstyledefinition}, \helpref{wxRichTextListStyleDefinition}{wxrichtextliststyledefinition} and \helpref{wxRichTextCharacterStyleDefinition}{wxrichtextcharacterstyledefinition}).
Once you have added definitions to a style sheet and associated it with a wxRichTextCtrl,
you can apply a named definition to a range of text. The classes \helpref{wxRichTextStyleComboCtrl}{wxrichtextstylecomboctrl}\rtfsp
and \helpref{wxRichTextStyleListBox}{wxrichtextstylelistbox} can be used to present the user with a list
of styles in a sheet, and apply them to the selected text.

You can reapply a style sheet to the contents of the control, by calling \helpref{wxRichTextCtrl::ApplyStyleSheet}{wxrichtextctrlapplystylesheet}.
This is useful if the style definitions have changed, and you want the content to reflect this.
It relies on the fact that when you apply a named style, the style definition name is recorded in the
content. So ApplyStyleSheet works by finding the paragraph attributes with style names and re-applying the definition's
attributes to the paragraph. Currently, this works with paragraph and list style definitions only.

\subsection{wxRichTextCtrl dialogs}\label{wxrichtextctrldialogs}

wxRichTextCtrl comes with standard dialogs to make it easier to implement
text editing functionality.

\helpref{wxRichTextFormattingDialog}{wxrichtextformattingdialog} can be used
for character or paragraph formatting, or a combination of both. It's a wxPropertySheetDialog
with the following available tabs: Font, Indents \& Spacing, Tabs, Bullets, Style, and List Style.
You can select which pages will be shown by supplying flags to the dialog constructor.
In a character formatting dialog, typically only the Font page will be shown.
In a paragraph formatting dialog, you'll show the Indents \& Spacing, Tabs and Bullets
pages. The Style tab is useful when editing a style definition.

You can customize this dialog by providing your own wxRichTextFormattingDialogFactory
object, which tells the formatting dialog how many pages are supported, what their identifiers
are, and how to creates the pages.

\helpref{wxRichTextStyleOrganiserDialog}{wxrichtextstyleorganiserdialog} is a multi-purpose dialog
that can be used for managing style definitions, browsing styles and applying them, or selecting list styles with
a renumber option. See the sample for usage - it is used for the "Manage Styles" and "Bullets and Numbering"
menu commands.

\helpref{wxSymbolPickerDialog}{wxsymbolpickerdialog} lets the user insert a symbol from
a specified font. It has no wxRichTextCtrl dependencies besides being included in
the rich text library.

\subsection{How wxRichTextCtrl is implemented}

Data representation is handled by wxRichTextBuffer, and a wxRichTextCtrl
always has one such buffer.

The content is represented by a hierarchy of objects, all derived from
wxRichTextObject. An object might be an image, a fragment of text, a paragraph,
or a whole buffer. Objects store a wxTextAttrEx containing style information;
a paragraph object can contain both paragraph and character information, but
content objects such as text can only store character information. The final
style displayed in the control or in a printout is a combination of base
style, paragraph style and content (character) style.

The top of the hierarchy is the buffer, a kind of wxRichTextParagraphLayoutBox.
containing further wxRichTextParagraph objects, each of which can include text,
images and potentially other types of object.

Each object maintains a range (start and end position) measured
from the start of the main parent object.

When Layout is called on an object, it is given a size which the object
must limit itself to, or one or more flexible directions (vertical
or horizontal). So, for example, a centred paragraph is given the page
width to play with (minus any margins), but can extend indefinitely
in the vertical direction. The implementation of Layout caches the calculated
size and position.

When the buffer is modified, a range is invalidated (marked as requiring
layout), so that only the minimum amount of layout is performed.

A paragraph of pure text with the same style contains just one further
object, a wxRichTextPlainText object. When styling is applied to part of
this object, the object is decomposed into separate objects, one object
for each different character style. So each object within a paragraph always has
just one wxTextAttrEx object to denote its character style. Of course, this can
lead to fragmentation after a lot of edit operations, potentially leading
to several objects with the same style where just one would do. So
a Defragment function is called when updating the control's display, to ensure that
the minimum number of objects is used.

\subsection{wxRichTextCtrl roadmap}

\wxheading{Bugs}

This is an incomplete list of bugs.

\begin{itemize}\itemsep=0pt
\item Moving the caret up at the beginning of a line sometimes incorrectly positions the
caret.
\item As the selection is expanded, the text jumps slightly due to kerning differences between
drawing a single text string versus drawing several fragments separately. This could
be improved by using wxDC::GetPartialTextExtents to calculate exactly where the separate fragments
should be drawn. Note that this problem also applies to separation of text fragments due to difference in their attributes.
\end{itemize}

\wxheading{Features}

This is a list of some of the features that have yet to be implemented. Help with them will be appreciated.

\begin{itemize}\itemsep=0pt
\item RTF input and output
\item Conversion from HTML
\item Open Office input and output
\item Floating images, with content wrapping around them
\item A ruler control
\item Standard editing toolbars
\item Tables
\item Bitmap bullets
\item Borders
\item Text frames
\item Justified text, in print/preview at least
\end{itemize}

There are also things that could be done to take advantage of the underlying text capabilities of the platform;
higher-level text formatting APIs are available on some platforms, such as Mac OS X, and some of translation from
high level to low level wxDC API is unnecessary. However this would require additions to the wxWidgets API.

