\section{\class{wxPropertySheetDialog}}\label{wxpropertysheetdialog}

This class represents a property sheet dialog: a tabbed dialog
for showing settings. It is optimized to show flat tabs
on PocketPC devices, and can be customized to use different
controllers instead of the default notebook style.

To use this class, call \helpref{wxPropertySheetDialog::Create}{wxpropertysheetdialogcreate} from your own
Create function. Then call \helpref{CreateButtons}{wxpropertysheetdialogcreatebuttons}, and create pages, adding them to the book control.
Finally call \helpref{LayoutDialog}{wxpropertysheetdialoglayoutdialog}.

For example:

\begin{verbatim}
bool MyPropertySheetDialog::Create(...)
{
    if (!wxPropertySheetDialog::Create(...))
        return false;

    CreateButtons(wxOK|wxCANCEL|wxHELP);

    // Add page
    wxPanel* panel = new wxPanel(GetBookCtrl(), ...);
    GetBookCtrl()->AddPage(panel, wxT("General"));

    LayoutDialog();
    return true;
}
\end{verbatim}

If necessary, override CreateBookCtrl and AddBookCtrl to create and add a different
kind of book control. You would then need to use two-step construction for the dialog.
Or, change the style of book control by calling \helpref{SetSheetStyle}{wxpropertysheetdialogsetsheetstyle} 
before calling Create.

The dialogs sample shows this class being used with notebook and toolbook controllers (for
Windows-style and Mac-style settings dialogs).

\wxheading{Derived from}

\helpref{wxDialog}{wxdialog}\\
\helpref{wxWindow}{wxwindow}\\
\helpref{wxEvtHandler}{wxevthandler}\\
\helpref{wxObject}{wxobject}

\wxheading{Include files}

<wx/propdlg.h>
<wx/generic/propdlg.h>

\latexignore{\rtfignore{\wxheading{Members}}}

\membersection{wxPropertySheetDialog::wxPropertySheetDialog}\label{wxpropertysheetdialogctor}

\func{}{wxPropertySheetDialog}{\param{wxWindow* }{parent}, \param{wxWindowID }{id},\rtfsp
\param{const wxString\& }{title},\rtfsp
\param{const wxPoint\& }{pos = wxDefaultPosition},\rtfsp
\param{const wxSize\& }{size = wxDefaultSize},\rtfsp
\param{long}{ style = wxDEFAULT\_DIALOG\_STYLE},\rtfsp
\param{const wxString\& }{name = ``dialogBox"}}

Constructor.

\membersection{wxPropertySheetDialog::AddBookCtrl}\label{wxpropertysheetdialogaddbookctrl}

\func{virtual void}{AddBookCtrl}{\param{wxSizer* }{sizer}}

Override this if you wish to add the book control in a way different from the
standard way (for example, using different spacing).

\membersection{wxPropertySheetDialog::Create}\label{wxpropertysheetdialogcreate}

\func{bool}{Create}{\param{wxWindow* }{parent}, \param{wxWindowID }{id},\rtfsp
\param{const wxString\& }{title},\rtfsp
\param{const wxPoint\& }{pos = wxDefaultPosition},\rtfsp
\param{const wxSize\& }{size = wxDefaultSize},\rtfsp
\param{long}{ style = wxDEFAULT\_DIALOG\_STYLE},\rtfsp
\param{const wxString\& }{name = ``dialogBox"}}

Call this from your own Create function, before adding buttons and pages.

\membersection{wxPropertySheetDialog::CreateBookCtrl}\label{wxpropertysheetdialogcreatebookctrl}

\func{virtual wxBookCtrlBase*}{CreateBookCtrl}{\void}

Override this if you wish to create a different kind of book control; by default, the value
passed to \helpref{SetSheetStyle}{wxpropertysheetdialogsetsheetstyle} is used to determine the control.
The default behaviour is to create a notebook except on Smartphone, where a choicebook is used.

\membersection{wxPropertySheetDialog::CreateButtons}\label{wxpropertysheetdialogcreatebuttons}

\func{void}{CreateButtons}{\param{int }{flags=wxOK|wxCANCEL}}

Call this to create the buttons for the dialog. This calls \helpref{wxDialog::CreateButtonSizer}{wxdialogcreatebuttonsizer}, and
the flags are the same. On PocketPC, no buttons are created.

\membersection{wxPropertySheetDialog::GetBookCtrl}\label{wxpropertysheetdialoggetbookctrl}

\constfunc{wxBookCtrlBase*}{GetBookCtrl}{\void}

Returns the book control that will contain your settings pages.

\membersection{wxPropertySheetDialog::GetInnerSizer}\label{wxpropertysheetdialoggetinnersizer}

\constfunc{wxSizer*}{GetInnerSizer}{\void}

Returns the inner sizer that contains the book control and button sizer.

\membersection{wxPropertySheetDialog::GetSheetStyle}\label{wxpropertysheetdialoggetsheetstyle}

\constfunc{long}{GetSheetStyle}{\void}

Returns the sheet style. See \helpref{SetSheetStyle}{wxpropertysheetdialogsetsheetstyle} for
permissable values.

\membersection{wxPropertySheetDialog::LayoutDialog}\label{wxpropertysheetdialoglayoutdialog}

\func{void}{LayoutDialog}{\param{int}{ centreFlags=wxBOTH}}

Call this to lay out the dialog. On PocketPC, this does nothing, since the dialog will be shown
full-screen, and the layout will be done when the dialog receives a size event.

\membersection{wxPropertySheetDialog::SetBookCtrl}\label{wxpropertysheetdialogsetbookctrl}

\func{void}{SetBookCtrl}{\param{wxBookCtrlBase* }{bookCtrl}}

Sets the book control used for the dialog. You will normally not need to use this.

\membersection{wxPropertySheetDialog::SetInnerSizer}\label{wxpropertysheetdialogsetinnersizer}

\func{void}{SetInnerSizer}{\param{wxSizer*}{ sizer}}

Sets the inner sizer that contains the book control and button sizer. You will normally not need to use this.

\membersection{wxPropertySheetDialog::SetSheetStyle}\label{wxpropertysheetdialogsetsheetstyle}

\func{void}{SetSheetStyle}{\param{long}{ style}}

You can customize the look and feel of the dialog by setting the sheet style. It is
a bit list of the following values:

\twocolwidtha{5cm}
\begin{twocollist}\itemsep=0pt
\twocolitem{wxPROPSHEET\_DEFAULT}{Uses the default look and feel for the controller window,
normally a notebook except on Smartphone where a choice control is used.}
\twocolitem{wxPROPSHEET\_NOTEBOOK}{Uses a notebook for the controller window.}
\twocolitem{wxPROPSHEET\_TOOLBOOK}{Uses a toolbook for the controller window.}
\twocolitem{wxPROPSHEET\_CHOICEBOOK}{Uses a choicebook for the controller window.}
\twocolitem{wxPROPSHEET\_LISTBOOK}{Uses a listbook for the controller window.}
\twocolitem{wxPROPSHEET\_TREEBOOK}{Uses a treebook for the controller window.}
\twocolitem{wxPROPSHEET\_SHRINKTOFIT}{Shrinks the dialog window to fit the currently selected page (common behaviour for
property sheets on Mac OS X).}
\end{twocollist}

