\documentclass[a4paper]{article}
\begin{document}


\title{The rsync algorithm}

\author{Andrew Tridgell \quad\quad Paul Mackerras\\
Department of Computer Science \\
Australian National University \\
Canberra, ACT 0200, Australia}

\maketitle

\begin{abstract}
  This report presents an algorithm for updating a file on one machine
  to be identical to a file on another machine.  We assume that the
  two machines are connected by a low-bandwidth high-latency
  bi-directional communications link.  The algorithm identifies parts
  of the source file which are identical to some part of the
  destination file, and only sends those parts which cannot be matched
  in this way.  Effectively, the algorithm computes a set of
  differences without having both files on the same machine.  The
  algorithm works best when the files are similar, but will also
  function correctly and reasonably efficiently when the files are
  quite different.
\end{abstract}

\section{The problem}

Imagine you have two files, $A$ and $B$, and you wish to update $B$ to be
the same as $A$. The obvious method is to copy $A$ onto $B$.

Now imagine that the two files are on machines connected by a slow
communications link, for example a dialup IP link.  If $A$ is large,
copying $A$ onto $B$ will be slow.  To make it faster you could
compress $A$ before sending it, but that will usually only gain a
factor of 2 to 4.

Now assume that $A$ and $B$ are quite similar, perhaps both derived
from the same original file. To really speed things up you would need
to take advantage of this similarity. A common method is to send just
the differences between $A$ and $B$ down the link and then use this
list of differences to reconstruct the file.

The problem is that the normal methods for creating a set of
differences between two files rely on being able to read both files.
Thus they require that both files are available beforehand at one end
of the link.  If they are not both available on the same machine,
these algorithms cannot be used (once you had copied the file over,
you wouldn't need the differences).  This is the problem that rsync
addresses.

The rsync algorithm efficiently computes which parts of a source file
match some part of an existing destination file.  These parts need not
be sent across the link; all that is needed is a reference to the part
of the destination file.  Only parts of the source file which are not
matched in this way need to be sent verbatim.  The receiver can then
construct a copy of the source file using the references to parts of
the existing destination file and the verbatim material.

Trivially, the data sent to the receiver can be compressed using any
of a range of common compression algorithms, for further speed
improvements.

\section{The rsync algorithm}

Suppose we have two general purpose computers $\alpha$ and $\beta$.
Computer $\alpha$ has access to a file $A$ and $\beta$ has access to
file $B$, where $A$ and $B$ are ``similar''.  There is a slow
communications link between $\alpha$ and $\beta$.

The rsync algorithm consists of the following steps:

\begin{enumerate}
\item $\beta$ splits the file $B$ into a series of non-overlapping
  fixed-sized blocks of size S bytes\footnote{We have found that
  values of S between 500 and 1000 are quite good for most purposes}.
  The last block may be shorter than $S$ bytes.

\item For each of these blocks $\beta$ calculates two checksums:
  a weak ``rolling'' 32-bit checksum (described below) and a strong
  128-bit MD4 checksum.

\item $\beta$ sends these checksums to $\alpha$.
  
\item $\alpha$ searches through $A$ to find all blocks of length $S$
  bytes (at any offset, not just multiples of $S$) that have the same
  weak and strong checksum as one of the blocks of $B$. This can be
  done in a single pass very quickly using a special property of the
  rolling checksum described below.
  
\item $\alpha$ sends $\beta$ a sequence of instructions for
  constructing a copy of $A$.  Each instruction is either a reference
  to a block of $B$, or literal data.  Literal data is sent only for
  those sections of $A$ which did not match any of the blocks of $B$.
\end{enumerate}

The end result is that $\beta$ gets a copy of $A$, but only the pieces
of $A$ that are not found in $B$ (plus a small amount of data for
checksums and block indexes) are sent over the link. The algorithm
also only requires one round trip, which minimises the impact of the
link latency.

The most important details of the algorithm are the rolling checksum
and the associated multi-alternate search mechanism which allows the
all-offsets checksum search to proceed very quickly. These will be
discussed in greater detail below.

\section{Rolling checksum}

The weak rolling checksum used in the rsync algorithm needs to have
the property that it is very cheap to calculate the checksum of a
buffer $X_2 .. X_{n+1}$ given the checksum of buffer $X_1 .. X_n$ and
the values of the bytes $X_1$ and $X_{n+1}$.

The weak checksum algorithm we used in our implementation was inspired
by Mark Adler's adler-32 checksum.  Our checksum is defined by
$$ a(k,l) = (\sum_{i=k}^l X_i) \bmod M $$
$$ b(k,l) = (\sum_{i=k}^l (l-i+1)X_i) \bmod M $$
$$ s(k,l) = a(k,l) + 2^{16} b(k,l) $$

where $s(k,l)$ is the rolling checksum of the bytes $X_k \ldots X_l$.
For simplicity and speed, we use $M = 2^{16}$.  

The important property of this checksum is that successive values can
be computed very efficiently using the recurrence relations

$$ a(k+1,l+1) = (a(k,l) - X_k + X_{l+1}) \bmod M $$
$$ b(k+1,l+1) = (b(k,l) - (l-k+1) X_k + a(k+1,l+1)) \bmod M $$

Thus the checksum can be calculated for blocks of length S at all
possible offsets within a file in a ``rolling'' fashion, with very
little computation at each point.

Despite its simplicity, this checksum was found to be quite adequate as
a first-level check for a match of two file blocks.  We have found in
practice that the probability of this checksum matching when the
blocks are not equal is quite low.  This is important because the much
more expensive strong checksum must be calculated for each block where
the weak checksum matches.

\section{Checksum searching}

Once $\alpha$ has received the list of checksums of the blocks of $B$,
it must search $A$ for any blocks at any offset that match the
checksum of some block of $B$.  The basic strategy is to compute the
32-bit rolling checksum for a block of length $S$ starting at each
byte of $A$ in turn, and for each checksum, search the list for a
match.  To do this our implementation uses a
simple 3 level searching scheme.

The first level uses a 16-bit hash of the 32-bit rolling checksum and
a $2^{16}$ entry hash table.  The list of checksum values (i.e., the
checksums from the blocks of $B$) is sorted according to the 16-bit
hash of the 32-bit rolling checksum.  Each entry in the hash table
points to the first element of the list for that hash value, or
contains a null value if no element of the list has that hash value.

At each offset in the file the 32-bit rolling checksum and its 16-bit
hash are calculated.  If the hash table entry for that hash value is
not a null value, the second-level check is invoked.

The second-level check involves scanning the sorted checksum list
starting with the entry pointed to by the hash table entry, looking
for an entry whose 32-bit rolling checksum matches the current value.
The scan terminates when it reaches an entry whose 16-bit hash
differs.  If this search finds a match, the third-level check is
invoked.

The third-level check involves calculating the strong checksum for the
current offset in the file and comparing it with the strong checksum
value in the current list entry.  If the two strong checksums match,
we assume that we have found a block of $A$ which matches a block of
$B$.  In fact the blocks could be different, but the probability of
this is microscopic, and in practice this is a reasonable assumption.

When a match is found, $\alpha$ sends $\beta$ the data in $A$ between
the current file offset and the end of the previous match, followed by
the index of the block in $B$ that matched.  This data is sent
immediately a match is found, which allows us to overlap the
communication with further computation.

If no match is found at a given offset in the file, the rolling
checksum is updated to the next offset and the search proceeds.  If a
match is found, the search is restarted at the end of the matched
block.  This strategy saves a considerable amount of computation for
the common case where the two files are nearly identical.  In
addition, it would be a simple matter to encode the block indexes as
runs, for the common case where a portion of $A$ matches a series of
blocks of $B$ in order.

\section{Pipelining}

The above sections describe the process for constructing a copy of one
file on a remote system.  If we have a several files to copy, we can
gain a considerable latency advantage by pipelining the process.

This involves $\beta$ initiating two independent processes. One of the
processes generates and sends the checksums to $\alpha$ while the
other receives the difference information from $\alpha$ and
reconstructs the files. 

If the communications link is buffered then these two processes can
proceed independently and the link should be kept fully utilised in
both directions for most of the time. 

\section{Results}

To test the algorithm, tar files were created of the Linux kernel
sources for two versions of the kernel. The two kernel versions were
1.99.10 and 2.0.0. These tar files are approximately 24MB in size and
are separated by 5 released patch levels.

Out of the 2441 files in the 1.99.10 release, 291 files had changed in
the 2.0.0 release, 19 files had been removed and 25 files had been
added.

A ``diff'' of the two tar files using the standard GNU diff utility
produced over 32 thousand lines of output totalling 2.1 MB.

The following table shows the results for rsync between the two files
with a varying block size.\footnote{All the tests in this section were
  carried out using rsync version 0.5}

\vspace*{5mm}
\begin{tabular}{|l|l|l|l|l|l|l|} \hline
{\bf block} & {\bf matches} & {\bf tag}  & {\bf false}  & {\bf data} & {\bf written}  & {\bf read} \\
{\bf size}  &               & {\bf hits} & {\bf alarms} &            &                &            \\ \hline \hline

300         & 64247         & 3817434    & 948          & 5312200    & 5629158        & 1632284   \\ \hline
500         & 46989         & 620013     & 64           & 1091900    & 1283906        & 979384    \\ \hline
700         & 33255         & 571970     & 22           & 1307800    & 1444346        & 699564    \\ \hline
900         & 25686         & 525058     & 24           & 1469500    & 1575438        & 544124    \\ \hline
1100        & 20848         & 496844     & 21           & 1654500    & 1740838        & 445204    \\ \hline
\end{tabular}
\vspace*{5mm}

In each case, the CPU time taken was less than the
time it takes to run ``diff'' on the two files.\footnote{The wall
  clock time was approximately 2 minutes per run on a 50 MHz SPARC 10
  running SunOS, using rsh over loopback for communication.  GNU diff
  took about 4 minutes.}

The columns in the table are as follows:

\begin{description}
\item [block size] The size in bytes of the checksummed blocks.
\item [matches] The number of times a block of $B$ was found in $A$.
\item [tag hits] The number of times the 16-bit hash of the rolling
  checksum matched a hash of one of the checksums from $B$.
\item [false alarms] The number of times the 32-bit rolling checksum
  matched but the strong checksum didn't.
\item [data] The amount of file data transferred verbatim, in bytes.
\item [written] The total number of bytes written by $\alpha$,
  including protocol overheads. This is almost all file data.
\item [read] The total number of bytes read by $\alpha$, including
  protocol overheads. This is almost all checksum information.
\end{description}

The results demonstrate that for block sizes above 300 bytes, only a
small fraction (around 5\%) of the file was transferred. The amount
transferred was also considerably less than the size of the diff file
that would have been transferred if the diff/patch method of updating
a remote file was used.

The checksums themselves took up a considerable amount of space,
although much less than the size of the data transferred in each
case. Each pair of checksums consumes 20 bytes: 4 bytes for the
rolling checksum plus 16 bytes for the 128-bit MD4 checksum.

The number of false alarms was less than $1/1000$ of the number of
true matches, indicating that the 32-bit rolling checksum is quite
good at screening out false matches. 

The number of tag hits indicates that the second level of the
checksum search algorithm was invoked about once every 50
characters.  This is quite high because the total number of blocks in
the file is a large fraction of the size of the tag hash table. For
smaller files we would expect the tag hit rate to be much closer to
the number of matches.  For extremely large files, we should probably
increase the size of the hash table.

The next table shows similar results for a much smaller set of files.
In this case the files were not packed into a tar file first. Rather,
rsync was invoked with an option to recursively descend the directory
tree. The files used were from two source releases of another software
package called Samba. The total source code size is 1.7 MB and the
diff between the two releases is 4155 lines long totalling 120 kB.

\vspace*{5mm}
\begin{tabular}{|l|l|l|l|l|l|l|} \hline
{\bf block} & {\bf matches} & {\bf tag}  & {\bf false}  & {\bf data} & {\bf written}  & {\bf read} \\
{\bf size}  &               & {\bf hits} & {\bf alarms} &            &                &            \\ \hline \hline

300         & 3727          & 3899       & 0            & 129775     & 153999         & 83948       \\ \hline
500         & 2158          & 2325       & 0            & 171574     & 189330         & 50908       \\ \hline
700         & 1517          & 1649       & 0            & 195024     & 210144         & 36828        \\ \hline
900         & 1156          & 1281       & 0            & 222847     & 236471         & 29048        \\ \hline
1100        & 921           & 1049       & 0            & 250073     & 262725         & 23988        \\ \hline
\end{tabular}
\vspace*{5mm}


\section{Availability}

An implementation of rsync which provides a convenient interface
similar to the common UNIX command rcp has been written and is
available for download from http://rsync.samba.org/

\end{document}
